\documentclass[titlepage, 13px, a4paper]{article}

\usepackage[utf8]{inputenc}

\usepackage[T1]{fontenc}
\usepackage{fontawesome}
\usepackage{eurosym}

\usepackage[french]{babel}

\usepackage{fancyhdr}
\usepackage{graphicx}
\usepackage[left=4cm,right=4cm,top=4cm,bottom=5cm,textheight=25cm]{geometry}
\usepackage{wrapfig}

\usepackage{eso-pic}
\usepackage{transparent}

\usepackage{hyperref}
\usepackage{setspace}

\usepackage{titletoc}

%\usepackage{titlesec}
%\titleformat{\part}[display]
  %{\normalfont\bfseries}{}{0pt}{\Large\bfseries}

\newcommand\BackgroundPic{%
	\put(0,-50){%
		\parbox[b][\paperheight]{\paperwidth}{%
			%\vfill
			\centering
			\includegraphics[%
			keepaspectratio]{../../images/icone.jpg}%
			\vfill
		}
	}
}

  
\renewcommand{\baselinestretch}{1.15}
\renewcommand{\partname}{}

\title{\textbf{{\Huge Rapport de soutenance \no 2}}}
\author{
	\\
	\bsc{{\LARGE ACCEr}} \\ \\
	\bsc{ShadowMiner} \\ \\ \\
	\bsc{CLAUDEL Antoine} \\
	\bsc{FARINAZZO Cédric} \\
	\bsc{LANGUERRE Clément} \\
	\bsc{GRIZZI Edgar} \\ \\ \\
}
\date{{\LARGE \today}}

\pagestyle{fancy}
\fancyfoot[L]{\includegraphics{../../images/ShadowMiners_logo_mini.png}} %50x33
\fancyhead[R]{ACCEr}
\fancyhead[L]{Rapport de soutenance \no 2}

\begin{document}
\AddToShipoutPicture*{\BackgroundPic}

\maketitle
\tableofcontents




\newpage 
\section{Introduction} 
\paragraph{} \hspace{0pt} 
Comme chaque année à Epita, les groupes doivent créer un rapport de soutenance pour chaque soutenance. \\
Dès la validation de notre cahier des charges nous avons imaginé nos mineurs prendre vie ainsi que le naissance du vil "Shadow Miner", monstre de notre jeu et ennemi de nos ou notre mineur(s).
Ce projet a pour but de réaliser un jeu vidéo. \\
Notre projet est divisé en trois soutenances distinctes. Nous réalisons notre projet avec Unity 3D et en c\#.
Nous allons vous présenter le rapport de notre première soutenance de ce projet décrivant notre premier pas dans le monde de la programmation et de jeu vidéo 
ainsi nous retracerons l'avancée de notre projet depuis la validation de notre cahier des charges qui a eu lieu il y a environ un mois et demi. 
Notre projet est constitué d'Edgar Grizzi, Clément Languerre, Cédric farinazzo et Antoine Claudel.
Pour vous rappeler un petit peu, notre jeu est un jeu aventure survie à l'intérieur d'une mine ou notre héros un mineur doit s'échapper d'une mine 
alors qu'il est poursuivi par un monstre, Le shadow miner.
Notre jeu est constitué d'un mode solo composé d'une série de niveau et d'un mode multijoueurs où 3 joueurs s'affrontent : 
1 Shadow Miner et 2 mineurs dont un ayant la possibilité de devenir un esprit de la mine pour prendre son contrôle.




\section{Contexte} 
\paragraph{} \hspace{0pt} 
Ce projet est réalisé dans le cadre de la mise en oeuvre des connaissances acquises 
lors des cours de TD et TP informatiques et nous permet de renforcer et d’approfondir nos apprentissages.   
Actuellement, notre groupe est sur le point de passer la première soutenance.

\newpage





\part{Avancement générale}  
\section{Les objectifs remplis} 
\paragraph{} \hspace{0pt} 
Entre cette soutenance et la précèdent, nous avons eu plus de temps. Nous avons pu bien avancer notre jeu. \\
Il a bien évolué depuis la première soutenance où nous avions juste 2 niveaux et un début de multijoueur.
Désormais, nous avons un beau menu, avec des niveaux solos, des paramètres fonctionnels et un mode multijoueur amélioré.

{\begin{enumerate}
	\item Nous avons créé quelques péfabs tels que des clés ou des pépites afin d’améliorer nos niveaux 
			et plonger dans un univers ressemblant le plus possible à une mine.	\\
	\item Nous avons amélioré notre site internet afin de pouvoir créer un pseudo pour chaque utilisateur qui est à bob par défaut.	\\
	\item Ainsi nous avons créé un serveur en C\# connecté à la base de données (MySql) du site web \url{https://accer.ddns.net/}
			et un client en C\# intégré au jeu. 
			Ce serveur et ce client communique entre eux sur le port 4247 et permettent à l’utilisateur de créer, se connecter ou
			manager son compte directement depuis le jeu. \\
			Ainsi les comptes crées depuis le jeu sont aussi accessibles sur le site avec les mêmes identifiants. \\
	\item Nous avons revu notre menu. Il comporte 6 boutons :
		{\begin{itemize}
			\item	Un bouton pour jouer au mode solo
			\item	Un bouton pour jouer au mode multijoueur
			\item	Un bouton pour ouvrir le site web dans un navigateur
			\item	Un bouton pour accéder au menu des comptes
			\item	Un bouton pour quitter le jeu \\
		\end{itemize}} 
	\item Un menu des paramètres a été ajouté afin de modifier la qualité des graphismes, la résolution de l’écran, 
		l’activation ou la du mode plein écran, changer la fréquence d’affichage, de régler la sensibilité de la souris, 
		de changer les touches, de changer le volume sonore de la musique et des effets sonores. \\
	\item Une série de scène permettant de créer, se connecter ou gérer son compte à été créer en lien avec le serveur C\#. \newpage
	\item En lien avec le menu des paramètres et les scènes permettent de gérer son compte, 
		nous avons ajouté une série de fonctionnalités permettant ainsi de recharger les données de l’utilisateur au lancement du jeu 
		et ainsi conserver ces paramètres ou encore sauvegardé le token et l’email permettant la reconnexion au serveur. \\
	\item Nous nous sommes rendus compte de freeze provoquer par le changement de scène. Nous avons donc une scène 
		nous permettant charger nos scènes de façons asynchrones et ainsi nous permettent de créer une barre d’avancement. \\
	\item Un menu des niveaux solo a été créer afin de rejoindre tous les niveaux solos. \\
	\item Le multiplayer a été revue afin de synchroniser les animations des joueurs et de faciliter la gestion des rooms.
		Ainsi lorsque qu’on se connecte au serveur multijoueur (Photon), une room est créé dans une scène multijoueur 
		choisi aléatoirement si aucune room n’est libre. \\
	\item Un menu pause a été ajouté au miner afin de permettre au joueur de faire une pause ou de retourner au menu. \\
	\item Une musique de fond et des bruits de pas ont été ajouté au jeu afin de combler le silence. \\
	\item Nous avons changé le modèle 3D du ShadowMiner monstre) afin de nous permettre de l’intégrer au jeu plus facilement.
		Une IA a été ajouté sur ce ShadowMiner afin de suivre un joueur lorsque ce dernier s’approche trop ou qu’il passe 
		dans le champ de vision du monstre. \\

\end{enumerate}}


\newpage




\section{Objectifs futurs jusque la soutenance finale} 
\paragraph{} \hspace{0pt} 
L'objectif pour la prochaine soutenance est en priorité de mettre une I.A sur le Shadow Miner, de plus pouvoir l'incarner 
dans le mode multijoueurs serait une belle avancée pour nous ainsi que pouvoir incarner la mine elle-même aussi serait 
un bel accomplissement pour nous. \\

Nous devrons remplir notre objectif de pouvoir rendre de meilleurs graphismes grâce à la modélisation 3D. \\

%Nous avons comme objectif de finaliser le script C\# animation ainsi que le script C\# joueur. 
Un des autres principaux objectifs est la cinématique qui est quelque chose auquel nous tenons, que nous avons annoncé, 
alors nous avons l'objectif de créer cette cinématique. \\

Certains sons ne sont pas encore mis dans le jeu comme le bruit du mineur quand il se blesse, lors de la dernière soutenance, 
La tâche du son du jeu sera remplie dans son intégralité. \\

Le menu sera entièrement finalisé avec un écran de fin de niveau. \\

Pour notre site internet et dans le jeu, une page de progression sera en lien avec notre jeu. \\

Pour notre serveur multijoueurs notre espérons l'rnvoi et la réception de la progression du joueur ainsi qu’afficher 
le pseudonyme du joueur au-dessus de celui-ci. \\

Ensuite nous aimerions créer un système qui permet de générer des maps aléatoirement pour le multijoueur qui pour 
le moment choisi une scène aléatoirement. \\

Le dernier objectif serait l'achat d'un CD-ROM, d'une jaquette pour compiler et enregistrer notre jeu sur un CD avec 
un potentiel livret de conseil d'utilisation. \\

Potentiellement, nous espérons pouvoir créer un éditeur de map et un trailer. \\


\newpage





\part{Quelques précisions ...}

\section{Précision sur le fonctionnement du mode multijoueurs}
\paragraph{} \hspace{0pt} \\
Nous avons souhaité ajouté quelques précisions sur le mode multijoueurs :
\\
{\begin{enumerate}
	\item Nous utilisons Photon Unity Networking pour le multijoueurs.
		\\
	\item Le joueur devra créer un compte pour pouvoir accéder au mode multijoueurs.
			Ce compte pourra être créer sur le site internet ou dans le jeu lui-même
			Ainsi le joueur pourra consulter et modifier ses données depuis le site internet ou le jeu.
			Ce système lui permet de jouer sur une version éxécutable du jeu comme sur la version web du jeu en conservant ses données.
			Ainsi la portabilité et l'accessibilité du jeu en seront augmentée.
		\\
	\item Etant donné le nombre de joueur pouvant être connecté au serveur (20 joueurs max), donc nous prévoyons 6 rooms de jeu avec des maps aléatoires.
		\\
\end{enumerate}}



\part{Apports collectifs et personnels}
\paragraph{} \hspace{0pt} \\
Lors de la première soutenance nous venions seulement de commencer à  programmer en groupe et l’excitation surpassait le reste . 
C'est seulement durant cette période post-soutenance que nous
nous sommes rendu compte des vraies difficultés d'un projet en groupe et que l'excitation s’est atténuée. \\

\paragraph{} \hspace{0pt} \\
Les points négatifs :
{\begin{itemize}
	\item	Dans un groupe chacun possède sa vision du projet et même si l'on finit toujours par trouver un compromis il arrive parfois que 
		certaines idées passent à la trappe ou qu'au contraire elles soient mises en place sans l'unanimité des voix.
		Certains parties réalisées par différentes personnes ne sont pas compatibles entre-elles et il faut les modifier voire totalement 
		les effacer puis les recréer. \\
		Tout le monde n'est pas d'accord sur les périodes de travail, or des créneaux de mise en commun
		et de prévision des tâches sont essentiels pour le groupe, contrairement au projets individuels.
		La fatigue entraînée par les longues heures de travail conduit parfois à des petits conflits ou des réflexions désagréables. \\
\end{itemize}} 

\paragraph{} \hspace{0pt} \\
Les points positifs :
{\begin{itemize}
	\item	Néanmoins le travail en groupe est très enrichissant pour chaque membre car en vérifiant ce que les autres ont produit 
		on en apprend beaucoup et on peut par la suite réutiliser ce savoir pour sa propre partie. \\
		Le travail en groupe permet de ne pas perdre un rythme de travail car personne ne souhaite être le poids mort du groupe. 
		Chacun donne donc du sien et fait des efforts pour faire progresser le projet. \\
		L’entraide est un des plus gros avantages du groupe et il a été utilisé de nombreuses fois. \\
		La répartition des tâches et aussi un point fort du travail en groupe et nous n'aurions sûrement 
		pas pu accomplir ce qu'on a fait jusqu'ici seul. \\
\end{itemize}} 






\newpage
\part{Problèmes rencontrées et sources }
\section{Problèmes rencontrées}
\paragraph{} \hspace{0pt} \\
Nous n'avons pas rencontré de problème majeurs mis à part que le network n'est pas fluide (nous trouverons une solution bientôt) 
et que lors d'un changement de scène la directionnel light ne s'active pas, du coup nous avons créé une seconde directionnel light pour patcher ce bug et ca marche !

\newpage




\part{Avancement Personnel}  
\section{Antoine}
\paragraph{} \hspace{0pt} \\
Depuis la dernière soutenance j'ai créé plusieurs niveaux. 
Au départ je construisais de nouveau niveaux à partir de rien et cela me prenait beaucoup de temps et d'effort pour au final 
obtenir quelque chose de très satisfaisant mais également trop rapide à jouer. Par la suite j'ai décidé de créer un seul niveau, 
assez garni, qui servira de modèle pour tous les autres et d'ajouter mes précédents « mini-niveaux » dans le modèle. 
Parmi ces anciens mini-niveaux il y avait un couloir enflammé qui empêche l'accès à la sortie et oblige le joueur à faire un détour, 
mais aussi un faux sol qui s’effondre lorsque le joueur passe dessus après avoir aperçu l'ennemi du jeu (le ShadowMiner). J'ai donc 
utilisé de nombreuses préfabs déjà construites lors de la première soutenance comme les murs et les portes, et de nouvelles préfabs 
comme les pépites dorées et les couloirs (les pépites sont faîtes à partir des cailloux construits sur Blender). Étant responsable du 
menu du jeu j'ai construit plusieurs interfaces utilisateur, une pour la connexion à son compte, une pour l'inscription, une pour le 
choix des niveaux et une pour l'affichage et l’édition des informations personnelles d'un compte. Comme on peut s'y attendre lorsqu'on 
éalise son premier jeu vidéo, tout ne s'est pas passé comme prévu et j'ai souvent du modifier voire supprimer certains de mes ajouts. 
Cédric m'a notamment aidé pour le script en C\# des interactions avec le serveur ainsi que l'optimisation. Lors de cette période suivant 
la première soutenance j'ai appris que dans un projet, et d'autant plus en informatique, il y a beaucoup de façon d'arriver au même résultat. 
Cela peut mener à des tensions dans le groupe car chacun veut faire à sa manière mais cela permet aussi d'apprendre les limites de ses méthodes 
et d’en découvrir des nouvelles parfois plus optimales au final. Pour la prochaine soutenance je me fixe comme objectifs de donner
 une touche personnelle à chaque niveau afin que le joueur ne trouve pas le jeu répétitif. Je devrai parallèlement tenir compte 
 du scénario pour produire des niveaux en rapport avec l'histoire du jeu.


\newpage





\section{Cédric}
\paragraph{} \hspace{0pt} \\
Dès le commencement du projet, j’ai beaucoup travaillé pour trouver des idées pour le groupe.
J’ai appris rapidement à utiliser \LaTeX afin de rédiger le cahier des charges.
Dès le début du projet, je me suis appliqué à la réalisation du site web en PHP. 
Etant donné le fait que les offres d’hébergements ne nous convenaient pas, 
nous avons décidé de l’héberger nous-même ! Ainsi j’héberge le site internet sur un Raspberry PI 3. 
Je n’ai pas rencontré de problèmes mis à part qu’il a fallu trouver une solution pour faire tenir le câble Ethernet …

\paragraph{} \hspace{0pt} \\
Dès la fin du cahier des charges, j’ai créé le projet sous Unity et ai commencé à travailler.
J’ai tous d’abord commencé à organiser notre projet en créant des dossiers pour ranger nos scripts, 
nos modèles 3D, nos textures, nos scènes …
J’ai ensuite créé une porte avec son animation. J’ai eu un peu de mal à trouver comment fonctionnait 
le menu Animation d’Unity mais je me suis vite habitué et j’ai pu montrer comment créer une animation 
aux autres membres du groupe. J’ai ensuite créé un script pour ouvrir et fermer la porte : 
il suffisait de jouer l’animation pour permettre ces opérations.
Grâce à un collider sphérique attaché à la porte avec IsTrigger coché j’ai pu détecter 
si un joueur était dans ce collider (OnTriggerStay()) et donc si il pressait la touche E j’activait la dite porte.
Ensuite je me suis appliqué à la création du mineur puisque Edgar ne peut faire fonctionner Unity 
et que Clément avait un problème d’écran d’ordinateur. Le plus dur a été de trouver notre mineur !
La création du script pour le contrôle des mouvements était relativement simple puisque des fonctions 
comme Input.GetKey() permettent de savoir si la touche passée en paramètre est pressée !
Le script et le mineur créé, j’ai créé 2 joueurs :
{\begin{itemize}
	\item Le premier avec une vue à la troisième personne qui servira pour les cinématiques et quelques niveaux
	\item Un autre en vue à la première personne pour les niveaux  
\end{itemize}}

\paragraph{} \hspace{0pt} \\
J’ai ensuite commencé à m’intéresser au Network et j’ai installé Photon Unity Networking, 
J’ai ensuite tenté de créer le script du Network Manger. Après quelques heures de doute, 
nous avons pu jouer à plusieurs ! Bien que les animations ne soient pas synchronisées et que le rendu visuel des autres joueurs est un peu décalé et saccadé, cela fonctionne et nous réglerons ces problèmes plus tard.
Etant chef de projet, j’ai dû attribuer des tâches à chacun. 
Les troupes ont eu un peu de retard à l’allumage mais ils y sont mis vraiment début mai 
même si certains cherchaient toujours une issue pour éviter les tâches que je leur donnais.
Mais finalement, j’ai réussi à faire en sorte que nous soyons dans les temps !
J’ai aussi beaucoup aidé mes collègues à utiliser Unity ou à comprendre comment fonctionne les scripts C\# dans Unity.
\\
Cette première partie du projet m’as permis de découvrir le rôle de chef de projet.


\begin{center}
	\begin{figure}[!hp]
	   \includegraphics[width=10cm]{networkcs-game.png}
	\end{figure}
	Le script permmetant de se connecter au serveur multijoueur.
\end{center}


\newpage





\section{Clément}
\paragraph{} \hspace{0pt} \\
A la suite de notre première soutenance, nous avions décidé d'organiser une séance par semaine afin d'avancer efficacement dans notre projet.
 On a pu alors parler des tâches que chacun devait accomplir et parler des mécaniques des prochains niveaux. Le premier niveau que j'avais 
 confectionné était un niveau pour débutant, sans réel intérêt mis à part prendre le jeu en main, découvrir le style de jeu et des mécaniques 
 basiques tels qu'ouvrir une porte, courir ou pousser un rocher pour débloquer un chemin. Par la suite, avec Antoine, nous souhaitions 
 rendre les niveaux plus complexes, avec des petites diffcultés afin de rendre le jeu plus attrayant. Ainsi, mon deuxième fait 
 apparaître le joueur dans une salle qui semblait être une ancienne salle de restauration (ancienne car la mine est abandonnée!). 
 Deux chemins s'offrent à lui, dont un qui lui sera bloqué par des chariots... Vides. Le joueur doit alors emprunter l'autre chemin. 
 Une porte fermé s'offre à lui et doit alors trouver les clefs qui sont dans une autre salle secrète qui se dévoile en débloquant un 
 passage secret. Ce passage secret est cachée par une amoire qui bouge quand on s'approche de cette dernière et en appuyant sur la 
 touche "E" du clavier. J'ai repris le préfab armoir que j'avais confectionné et lui ai implémenté deux animations ainsi qu'un script C\#. 
 Malheureusement, la mécanique de récupérer le trousseau de clé et le stocker dans un inventaire n'a pas encore été faite. 
 Ainsi, le niveau n'est pas terminé à 100%. 
Pour le troisième niveau, je voulais commencé à introduir le personnage emblématique du jeu, j'ai nommé le "Shadow Miner",
controlé par un IA et qui se déplacerait entre deux points  de la map et qui nous poursuivrai s'il nous détecterait. 
Le personnage commence dans un des nombreux couloir de cette mine abandonnée et doit aller à l'autre bout. 
Problème? une grille barre son passage. Il doit alors activer un levier qui se trouve dans le niveau. 
Mais gare au monstre de la mine qui rode! 
Le niveau n'est pas à 100% de ces capacités. L'animation du leveir qui s'abaisse et qui ouvrirait la grille n'est pas encore faite. 
J'ai créé de nouveaux préfabs comme la clef, le kit "Table + Tabourets" et la grille. Pour la prochaine soutenance, 
je souhaiterai finir toutes les mécaniques du deuxième et troisième niveau ainsi que d'imaginer d'autres mécaniques qui rendraient
 le jeu plus complexe, travailler sur les cinématiques et sur la confection de nouveau niveau. 
Notre groupe est toujours aussi bien soudé, nous travaillons souvent ensemble, que ça soit au campus ou par Discord. 
J'aime particulièrement les moments où Antoine me fait tester ses niveaux et quand je fais tester mes niveaux au groupe; 
En testant les niveaux d'Antoine , celà me permet aussi d'imaginer d'autres niveaux, et vice-versa. 
Ce projet permet de nous épanouir psychologiquement, mais aussi à se répartir les tâches et à travailler en groupe.

\newpage





\section{Edgar}
\paragraph{} \hspace{0pt} \\
Entre la première et la seconde soutenance Edgar s’est occupé en particulier des sons du jeu. 
Il a recherché plusieurs sons pour notre personnage de mineur comme des bruits de pas quand il marche des bruit de pas quand il court, 
des bruits quand il se blesse ou encore quand il meurt. Il a aussi recherché des bruits correspondant au Shadow Miner. 
Certains pour lui n’était pas assez bon pour le jeu donc il en a enregistré deux avec son téléphone. 
D’autres son été inaccessible au niveau de la tonalité, un changement de tonalité s'imposait alors il a utilisé Audacity pour 
s’occuper justement des effets sonores pour qu’il y ait une bonne tonalité aux sons de notre jeu, pour qu’il soit d’une assez bonne 
vitesse pour correspondre aux animations. Il a aussi aidé à trouver une musique de fond pour le jeu qui a été composé par Nicolas 
Masciocchi qui nous donne généreusement sa musique. Cette musique ensuite été découpé pour être utilisé pour plusieurs endroits ou 
parties de jeu comme la fin d’un niveau ou la musique de fond dans le menu. 

Edgar s’est renseigné aussi pour la création d’un système qui permettrait de générer des maps aléatoirement sans résultats positifs. 

Il a particulièrement apprécié s'occuper du son du jeu car il trouve ça extrêmement intéressant et c’est pour ça qu’il était motivé 
pour obtenir les meilleurs résultats possibles et qui selon lui était meilleur pour le jeu comme des bruits de pas bien synchronisée 
à l’animation de notre mineur. Edgar est toujours très motivé pour le projet et montre que ce jeu lui tient à cœur, qu'il fait 
ce qu’il apprécie dans un groupe qui communique et travaille bien.







\newpage
\part{Organisation du projet}
\paragraph{} \hspace{0pt} \\ 
Nous avons jugé utile de mettre à jour ce tableau de répartition des tâches
\\ \\
{\normalsize
	\begin{tabular}{|p{6cm}|p{1.2cm}|p{1.2cm}|p{1.2cm}|p{1.2cm}|}
		\hline
		Tâches & \multicolumn{4}{|c|}{Personnes} \\ 
		\cline{2-5}
			& Antoine & Cédric & Clément & Edgar \\
		\hline
		Création des préfabs de map pour les niveaux (murs, sol, porte, ..) & Supp\footnotemark[2] & X & Resp\footnotemark[1] & X \\
		\hline
		Modélisation 3D pour de meilleurs graphismes (si possible) & Supp\footnotemark[2] & X & X & Resp\footnotemark[1] \\
		\hline
		Script c\# animation & X & Resp\footnotemark[1] & Supp\footnotemark[2] & X \\
		\hline
		Création des préfabs des joueurs & Supp\footnotemark[2] & Resp\footnotemark[1] & X &  \\
		\hline
		Script c\# joueur & X & Resp\footnotemark[1] & X & Supp\footnotemark[2] \\
		\hline
		Création de multiples niveaux (entre 20 et 40) & Resp\footnotemark[1] & X & Supp\footnotemark[2] & X \\
		\hline
		Création du Shadow Miner et Script c\# pour l'IA du Shadow Miner & X & Resp\footnotemark[1] & X & Supp\footnotemark[2] \\
		\hline
		Cinématique du jeu & X & X & Resp\footnotemark[1] & Supp\footnotemark[2] \\
		\hline
		Son du jeu & X & X & Supp\footnotemark[2] & Resp\footnotemark[1] \\
		\hline
		Menu du jeu & Resp\footnotemark[1] & X & Supp\footnotemark[2] & \\
		\hline
		Création du site internet et Hébergement en ligne & X & Resp\footnotemark[1] & X & Supp\footnotemark[2] \\
		\hline
		Création du serveur multijoueurs & Supp\footnotemark[2] & Resp\footnotemark[1] & X & X \\
		\hline
		Création des joueurs pour multijoueurs & Supp\footnotemark[2] & Resp\footnotemark[1] & X & X \\
		\hline
		Création du système de map aléatoire pour le multijoueurs & Supp\footnotemark[2] & X & X & Resp\footnotemark[1] \\
		\hline
		Compte rendu en \LaTeX & X & Resp\footnotemark[1] & X & Supp\footnotemark[2]  \\
		\hline
		Compilation du jeu et enregistrement sur CD & X & X & Resp\footnotemark[1] & Supp\footnotemark[2] \\
		\hline
		Création trailer , plaquette et manuel d'installation et d'utilisation du jeu & X & X & Supp\footnotemark[2] & Resp\footnotemark[1] \\
		\hline
	\end{tabular}
	\label{repartition}		
	\footnotetext[1]{Responsable}
	\footnotetext[2]{Suppléant}
}





\newpage
\part{Quelques sources ...}
\paragraph{} \hspace{0pt} \\ 
N'etant pas des experts en modélisation 3D, nous avons du trouvé nos personnages sur internet : 
{\begin{itemize}
	\item Le mineur provient des standarts assets de Unity 4.x, nous avons juste récupéré le modele 3D et les animations. 
	car nous n'avons pas le droit de récupérer de script et que les scripts de l'assets étaient en Javascript(Unityscript).
	\item les textures sur les objet et les sols proviennent de l'Assets Store.
\end{itemize}}






\newpage

\part{Conclusion}
Nous avons bien avancé et nous avons remplis nos objectifs.
Les personnage sont en place ainsi que 2 niveaux et le multijoueur.
\\ \\
Nous comptons donc continue dans cette voie en continuant le multijour, la création de préfabs, la modélisation 3D d'objets, 
la création du profile du joueur et de la sauvegarde de sa progression.
\\ \\
Ce projet nous tient tous à coeur et nous voulons nous dépasser afin d'obtenir ce que nous voulons ! 

\end{document}